Autonomous navigation systems are a rapidly advancing field, encompassing academic research, commercial products, and enthusiast prototypes. Research from the University of Michigan on LIDAR and visual sensor navigation \cite{Howard2023} provides robust solutions but tends to be too complex and resource-intensive for educational use. Similarly, ETH Zurich's sensor fusion and mapping systems \cite{Allen2022} primarily target advanced robotics, making them less accessible to undergraduate students.
\\\\
Enthusiast prototypes like the TurtleBot \cite{TurtleBot2021} offer a foundation for DIY autonomous systems but often lack the robustness and structure necessary for formal education. Commercial products such as the iRobot Roomba and Starship Technologies delivery robots \cite{Starship2022} effectively demonstrate indoor navigation but are closed-source, expensive, and challenging to modify for educational use.
\\\\
Educational robotics kits like LEGO Mindstorms and VEX Robotics \cite{LEGO2020} introduce students to basic concepts but do not cover the complexity of real-world autonomous navigation.
\\\\
The current solutions in the market whether academic, enthusiast driven, or commercially available do not fully address the needs of students seeking a comprehensive, hands-on education in autonomous navigation. Academic research is often too advanced and resource intensive, enthusiast prototypes lack educational integration, and commercially available products are either too expensive, too simplified, or proprietary. These limitations highlight the necessity of a custom built solution like the Roam\_Bot, which offers students the ability to work directly with real world navigation challenges in a controlled, affordable, and educationally driven platform.

%%%%%%%%%%%%%%%%%%%%%%%%%%%%%%%%%%%%%%%%%%%
%% INSTRUCTIONS
%%%%%%%%%%%%%%%%%%%%%%%%%%%%%%%%%%%%%%%%%%%

% Discuss the state-of-the-art with respect to your product. What solutions currently exist, and in what form (academic research, enthusiast prototype, commercially available, etc.)? Include references and citations as necessary using the \textit{\textbackslash cite} command, like this \cite{Rubin2012}. If there are existing solutions, why won't they work for your customer (too expensive, not fast enough, not reliable enough, etc.). This section should occupy 1/2 - 1 full page, and should include at least 5 references to related work. All references should be added to the \textit{.bib} file, fully documented in IEEE format, and should appear in the \textit{references} section at the end of this document (the IEEE citation style will automatically be applied if your reference is properly added to the \textit{.bib} file).

% ProTip: Consider using a citation manager such as Mendeley, Zotero, or EndNote to generate your \textit{.bib} file and maintain documentation references throughout the life cycle of the project.

%%%%%%%%%%%%%%%%%%%%%%%%
% To cite something, use the \cite command with the name of the bibtex entry in the curly braces.
% It will determine which reference number it is and insert that number where the \cite command is.
% e.g. \cite{Rubin2012}