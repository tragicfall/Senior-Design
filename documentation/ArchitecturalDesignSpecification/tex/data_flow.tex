% This section breaks down your layer abstraction to another level of detail. Here you grapically represent the logical subsytems that compose each layer and show the interactions/interfaces between those subsystems. A subsystem can be thought of as a programming unit that implements one of the major functions of the layer. It, therefore, has data elements that serve as source/sinks for other subsystems. The logical data elements that flow between subsystems need to be explicitly defined at this point, beginning with a data flow-like diagram based on the block diagram.

%%%%%%%%%%%%%%%%%%%%%%%%%%%%%%%%%%%%%%%%%%%%%%%%%%%%%%%%%%%%%%%%%%%%%%%%%%%%
%   Change the graphic here. Put your image in the 'images' folder
%   and update the name from 'images/test_image' to your image name
\begin{figure}[h!]
	\centering
 	\includegraphics[width=\textwidth]{images/data_flow_2nd.jpeg}
 \caption{Data Flow Diagram} % Be sure to change the caption!
\end{figure}
