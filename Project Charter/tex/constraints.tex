%Constraints are limitations imposed on the project, such as the limitation of cost, schedule, or resources, and you have to work within the boundaries restricted by these constraints. All projects have constraints, which are defined and identified at the beginning of the project.

%Constraints are outside of your control. They are imposed upon you by your client, organization, government regulations, availability of resources, etc. Occasionally, identified constraints turn out to be false. This is often beneficial to the development team, since it removes items that could potentially affect progress.

%This section should contain a list of at least 5 of the most critical constraints related to your project. For example:

The following list contains key constraints related to the implementation and testing of the project.

%%%%%%%%%%%%%%%%%%%%%%%%%%%%%%%%%%%%%%%%%%%%%%%%%%%%%%%%%%%%%%%%%%%%%%%%%
% This creates a bullet list. To add a bullet, use the \item command.
% Make sure it is between the \begin{itemize} and \end{itemize} commands.
% The indentation of the items is optional and is for code readability.
\begin{itemize}
  \item The base robot was not initially designed by our team.
  \item Limited to certain components that would be out of budget to replace.
  \item Limited range of motion due to design of robot wheels.
  \item Total development costs must not exceed \$800.
  \item Limited to the sensors, such as limited to the range of the LiDAR.
\end{itemize}
